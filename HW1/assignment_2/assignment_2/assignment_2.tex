%%%%%%%%%%%%%%%%%%%%%%%%%%%%%%%%%%%%%%%%%
% Programming/Coding Assignment
% LaTeX Template
%
% This template has been downloaded from:
% http://www.latextemplates.com
%
% Original author:
% Ted Pavlic (http://www.tedpavlic.com)
%
% Note:
% The \lipsum[#] commands throughout this template generate dummy text
% to fill the template out. These commands should all be removed when 
% writing assignment content.
%
% This template uses a Perl script as an example snippet of code, most other
% languages are also usable. Configure them in the "CODE INCLUSION 
% CONFIGURATION" section.
%
%%%%%%%%%%%%%%%%%%%%%%%%%%%%%%%%%%%%%%%%%

%----------------------------------------------------------------------------------------
%	PACKAGES AND OTHER DOCUMENT CONFIGURATIONS
%----------------------------------------------------------------------------------------

\documentclass{article}

\usepackage{fancyhdr} % Required for custom headers
\usepackage{lastpage} % Required to determine the last page for the footer
\usepackage{extramarks} % Required for headers and footers
\usepackage[usenames,dvipsnames]{color} % Required for custom colors
\usepackage{graphicx} % Required to insert images
\usepackage{listings} % Required for insertion of code
\usepackage{courier} % Required for the courier font
\usepackage{lipsum} % Used for inserting dummy 'Lorem ipsum' text into the template
\usepackage{amsmath,amssymb,amsthm}



% Margins
\topmargin=-0.45in
\evensidemargin=0in
\oddsidemargin=0in
\textwidth=6.5in
\textheight=9.0in
\headsep=0.25in

\linespread{1.1} % Line spacing

% Set up the header and footer
\pagestyle{fancy}
\lhead{\hmwkAuthorName} % Top left header
\chead{\hmwkClass\ (\hmwkClassInstructor\ \hmwkClassTime): \hmwkTitle} % Top center head
\rhead{\firstxmark} % Top right header
\lfoot{\lastxmark} % Bottom left footer
\cfoot{} % Bottom center footer
\rfoot{Page\ \thepage\ of\ \protect\pageref{LastPage}} % Bottom right footer
\renewcommand\headrulewidth{0.4pt} % Size of the header rule
\renewcommand\footrulewidth{0.4pt} % Size of the footer rule

\setlength\parindent{0pt} % Removes all indentation from paragraphs

%----------------------------------------------------------------------------------------
%	CODE INCLUSION CONFIGURATION
%----------------------------------------------------------------------------------------

\definecolor{MyDarkGreen}{rgb}{0.0,0.4,0.0} % This is the color used for comments
\lstloadlanguages{Perl} % Load Perl syntax for listings, for a list of other languages supported see: ftp://ftp.tex.ac.uk/tex-archive/macros/latex/contrib/listings/listings.pdf
\lstset{language=Perl, % Use Perl in this example
        frame=single, % Single frame around code
        basicstyle=\small\ttfamily, % Use small true type font
        keywordstyle=[1]\color{Blue}\bf, % Perl functions bold and blue
        keywordstyle=[2]\color{Purple}, % Perl function arguments purple
        keywordstyle=[3]\color{Blue}\underbar, % Custom functions underlined and blue
        identifierstyle=, % Nothing special about identifiers                                         
        commentstyle=\usefont{T1}{pcr}{m}{sl}\color{MyDarkGreen}\small, % Comments small dark green courier font
        stringstyle=\color{Purple}, % Strings are purple
        showstringspaces=false, % Don't put marks in string spaces
        tabsize=5, % 5 spaces per tab
        %
        % Put standard Perl functions not included in the default language here
        morekeywords={rand},
        %
        % Put Perl function parameters here
        morekeywords=[2]{on, off, interp},
        %
        % Put user defined functions here
        morekeywords=[3]{test},
       	%
        morecomment=[l][\color{Blue}]{...}, % Line continuation (...) like blue comment
        numbers=left, % Line numbers on left
        firstnumber=1, % Line numbers start with line 1
        numberstyle=\tiny\color{Blue}, % Line numbers are blue and small
        stepnumber=5 % Line numbers go in steps of 5
}

% Creates a new command to include a perl script, the first parameter is the filename of the script (without .pl), the second parameter is the caption
\newcommand{\perlscript}[2]{
\begin{itemize}
\item[]\lstinputlisting[caption=#2,label=#1]{#1.pl}
\end{itemize}
}

%----------------------------------------------------------------------------------------
%	DOCUMENT STRUCTURE COMMANDS
%	Skip this unless you know what you're doing
%----------------------------------------------------------------------------------------

% Header and footer for when a page split occurs within a problem environment
\newcommand{\enterProblemHeader}[1]{
\nobreak\extramarks{#1}{#1 continued on next page\ldots}\nobreak
\nobreak\extramarks{#1 (continued)}{#1 continued on next page\ldots}\nobreak
}

% Header and footer for when a page split occurs between problem environments
\newcommand{\exitProblemHeader}[1]{
\nobreak\extramarks{#1 (continued)}{#1 continued on next page\ldots}\nobreak
\nobreak\extramarks{#1}{}\nobreak
}

\setcounter{secnumdepth}{0} % Removes default section numbers
\newcounter{homeworkProblemCounter} % Creates a counter to keep track of the number of problems

\newcommand{\homeworkProblemName}{}
\newenvironment{homeworkProblem}[1][Problem \arabic{homeworkProblemCounter}]{ % Makes a new environment called homeworkProblem which takes 1 argument (custom name) but the default is "Problem #"
\stepcounter{homeworkProblemCounter} % Increase counter for number of problems
\renewcommand{\homeworkProblemName}{#1} % Assign \homeworkProblemName the name of the problem
\section{\homeworkProblemName} % Make a section in the document with the custom problem count
\enterProblemHeader{\homeworkProblemName} % Header and footer within the environment
}{
\exitProblemHeader{\homeworkProblemName} % Header and footer after the environment
}

\newcommand{\problemAnswer}[1]{ % Defines the problem answer command with the content as the only argument
\noindent\framebox[\columnwidth][c]{\begin{minipage}{0.98\columnwidth}#1\end{minipage}} % Makes the box around the problem answer and puts the content inside
}

\newcommand{\homeworkSectionName}{}
\newenvironment{homeworkSection}[1]{ % New environment for sections within homework problems, takes 1 argument - the name of the section
\renewcommand{\homeworkSectionName}{#1} % Assign \homeworkSectionName to the name of the section from the environment argument
\subsection{\homeworkSectionName} % Make a subsection with the custom name of the subsection
\enterProblemHeader{\homeworkProblemName\ [\homeworkSectionName]} % Header and footer within the environment
}{
\enterProblemHeader{\homeworkProblemName} % Header and footer after the environment
}

%----------------------------------------------------------------------------------------
%	NAME AND CLASS SECTION
%----------------------------------------------------------------------------------------

\newcommand{\hmwkTitle}{Assignment\ \#1} % Assignment title
\newcommand{\hmwkDueDate}{Thursday,\ January\ 25,\ 2018} % Due date
\newcommand{\hmwkClass}{MIS\ 381} % Course/class
\newcommand{\hmwkClassTime}{11: 00am} % Class/lecture time
\newcommand{\hmwkClassInstructor}{Kumar} % Teacher/lecturer
\newcommand{\hmwkAuthorName}{Korawat Tanwisuth} % Your name

%----------------------------------------------------------------------------------------
%	TITLE PAGE
%----------------------------------------------------------------------------------------

\title{
\vspace{2in}
\textmd{\textbf{\hmwkClass:\ \hmwkTitle}}\\
\normalsize\vspace{0.1in}\small{Due\ on\ \hmwkDueDate}\\
\vspace{0.1in}\large{\textit{\hmwkClassInstructor\ \hmwkClassTime}}
\vspace{3in}
}

\author{\textbf{\hmwkAuthorName}}
\date{} % Insert date here if you want it to appear below your name

%----------------------------------------------------------------------------------------

\begin{document}

\maketitle

%----------------------------------------------------------------------------------------
%	TABLE OF CONTENTS
%----------------------------------------------------------------------------------------

%\setcounter{tocdepth}{1} % Uncomment this line if you don't want subsections listed in the ToC


\newpage

%----------------------------------------------------------------------------------------
%	PROBLEM 1
%----------------------------------------------------------------------------------------

% To have just one problem per page, simply put a \clearpage after each problem

\begin{homeworkProblem}


We will first show that $(AB)^T= B^T A^T$

\begin{proof}

$((AB)^T)_{ij}  = (AB)_{ji} = \sum_{k=1}^{n}a_{jk}b_{ki}$

$(B^T A^T)_{ij} = \sum_{k=1}^{n}(B^T)_{ik} (A^T)_{kj} = \sum_{k=1}^{n}b_{ki}a_{jk}$

Thus, $(AB)^T= B^T A^T$
\end{proof}

Now, we will show that $(A^{-1})^T=(A^T)^{-1}$
\begin{proof}

$I^T=I$

$(AA^{-1})^T=I$

$(A^{-1})^T (A)^T=I$ since $(AB)^T= B^T A^T$

$(A^{-1})^T (A)^T(A^T)^{-1}=(A^T)^{-1}$

$(A^{-1})^T =(A^T)^{-1}$ 
\end{proof}




\end{homeworkProblem}
\newpage

%----------------------------------------------------------------------------------------
%	PROBLEM 2
%----------------------------------------------------------------------------------------

\begin{homeworkProblem}
Denote\\
$x_{1} = $Amount invested in the first mortgage\\
$x_{2} = $Amount invested in the second mortage\\
$x_{3} = $Amount invested in home improvement\\
$x_{4} = $Amount invested in personal overdraft\\

$x_{1}+x_{2}+x_{3}+x_{4} = 250\\
0.25x_{1}-0.75x_{2}+0.25x_{3}+0.25x_{4} = 0\\
-0.45x_{1}+0.55x_{2}= 0\\
0.14x_{1}+0.2x_{2}+0.2x_{3}+0.1x_{4} = 250*0.15=37.5\\
$

We will write this system of linear equations in matrix form.\\

Let

\[
A=
\begin{bmatrix}
1&1&1&1\\
0.25&-0.75&0.25&0.25\\
-0.45&0.55&0&0\\
0.14&0.2&0.2&0.1\\
\end{bmatrix}
\]
\[
A^{-1}=
\begin{bmatrix}
0.3055&-1.2222&-2.2222&0\\
0.25&-1&0&0\\
-1.372&1.4888& 0.8888&10\\
1.816&0.7333 &1.333&-10\\
\end{bmatrix}
\]
\[
x=
\begin{bmatrix}
x_{1}\\
x_{2}\\
x_{3}\\
x_{4}\\
\end{bmatrix}
\]
\[
b=
\begin{bmatrix}
250\\
0\\
0\\
37.5\\
\end{bmatrix}
\]

R-code: \\
$A = matrix(c(1,0.25,-0.45,0.14,1,-0.75,0.55,0.2,1,0.25,0,0.2,1,0.25,0,0.1),nrow=4)\\
b = matrix(c(250,0,0,37.5),nrow=4)\\
A\_inv = solve(A)\\
x = A\_inv\%*\%b
$

$ Ax = b$

$ x = A^{-1}b$
\[
x=
\begin{bmatrix}
76.3889\\
62.5\\
31.9444\\
79.1667\\
\end{bmatrix}
\]

\end{homeworkProblem}
\newpage
\begin{homeworkProblem}
Denote\\
$x_{i} = $ \# of units produced for variant i $\forall i \in \{1,2,3,4\}$


maximize:\\
$f=1.5x_{1}+2.5x_{2}+3x_{3}+4.5x_{4} $\\
subject to:\\
Assembly Constraint - $2x_{1}+4x_{2}+3x_{3}+7x_{4}\leq 100000$\\
Polish Constraint - $3x_{1}+2x_{2}+3x_{3}+4x_{4}\leq 50000$\\
Pack Constraint - $2x_{1}+3x_{2}+2x_{3}+5x_{4}\leq 60000$\\
Non-Negativity Constraint - $-x_{i}\leq 0 $ $\forall i \in \{1,2,3,4\}$\\

Let

\[
A=
\begin{bmatrix}
2&4&3&7\\
3&2&3&4\\
2&3&2&5\\
-1&0&0&0\\
0&-1&0&0\\
0&0&-1&0\\
0&0&0&-1\\
\end{bmatrix}
\]

\[
x=
\begin{bmatrix}
x_{1}\\
x_{2}\\
x_{3}\\
x_{4}\\
\end{bmatrix}
\]

\[
b=
\begin{bmatrix}
100000\\
50000\\
60000\\
0\\
0\\
0\\
0\\
\end{bmatrix}
\]

R-code:\\
$z = c(1.5,2.5,3,4.5)\\
A\_temp = matrix(c(2,3,2,4,2,3,3,3,2,7,4,5),nrow = 3)\\
I = -diag(4)\\
A = rbind(A\_temp,I)\\
b = c(100000,50000,60000,0,0,0,0)\\
signs = rep("<=",7)\\
ans = lp("max", z, A, signs, b)\\
$\\
\[
x=
\begin{bmatrix}
0\\
16000\\
6000\\
0\\
\end{bmatrix}
\]
The objective function is 58000. 
Thus, we would produce 16000 units of variant 2 and 6000 of variant 3 and should obtain a profit of 58000.
\end{homeworkProblem}
\newpage
\begin{homeworkProblem}
Denote\\

$r_{i} =$ rating of team i $\forall i \in \{1,2,3,4,5\}$\\

Let 

\[
A=
\begin{bmatrix}
1&-1&0&0&0\\
1&0&-1&0&0\\
1&0&0&-1&0\\
1&0&0&0&-1\\
0&1&-1&0&0\\
0&1&0&-1&0\\
0&1&0&0&-1\\
0&0&1&-1&0\\
0&0&1&0&-1\\
0&0&0&1&-1\\
1&1&1&1&1\\
\end{bmatrix}
\]

\[
r=
\begin{bmatrix}
r_{1}\\
r_{2}\\
r_{3}\\
r_{4}\\
r_{5}\\
\end{bmatrix}
\]


\[
b=
\begin{bmatrix}
-45\\
-3\\
-31\\
-45\\
18\\
8\\
20\\
2\\
-27\\
-38\\
0\\
\end{bmatrix}
\]\\

We will solve $Ar=b$. However, b is not in the column space of A so we will find $\hat{r}$ by projecting b into column space of A.\\

$A\hat{r}=b$\\
$A^TA\hat{r}=A^Tb$\\
$(A^TA)^{-1}(A^TA)\hat{r}=(A^TA)^{-1}A^Tb$\\
$\hat{r}=(A^TA)^{-1}A^Tb$\\



R-code:\\
$A= matrix(rep(0,50),nrow=10)\\
i = 1\\
k=4\\
l=k\\
j= 1\\
p = 1\\
for (r in 1:10)\{\\
    A[i,j] = 1\\
    A[i,j+p] = -1\\
    l=l-1\\
    i=i+1\\
    p = p+1\\
    if(l==0)\{\\
        j=j+1\\
        l=k-1\\
        k=k-1\\
        p = 1\\
    \}\\
\}\\
A = rbind(A,rep(1,5))\\
b = matrix(c(7-52,21-24,7-38,0-45,34-16,25-17,27-7,7-5,3-30,14-52,0),nrow=11)\\
A.T = t(A)\\
A.T\_A = A.T\%\*\%A\\
x = (solve(A.T\_A)\%\*\%A.T)\%\*\%b$\\

\[
\hat{r}=
\begin{bmatrix}
-24.8\\
18.2\\
-8.0\\
-3.4\\
18.0\\
\end{bmatrix}
\]\\

Thus, $r_{1} = -24.8, r_{2} = 18.2, r_{3} = -8.0, r_{4}=-3.4, r_{5} = 18.0$
\end{homeworkProblem}

%----------------------------------------------------------------------------------------

\end{document}